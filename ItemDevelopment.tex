% Options for packages loaded elsewhere
\PassOptionsToPackage{unicode}{hyperref}
\PassOptionsToPackage{hyphens}{url}
%
\documentclass[
]{article}
\title{ShareStats}
\usepackage{etoolbox}
\makeatletter
\providecommand{\subtitle}[1]{% add subtitle to \maketitle
  \apptocmd{\@title}{\par {\large #1 \par}}{}{}
}
\makeatother
\subtitle{Item Development}
\author{}
\date{\vspace{-2.5em}19 jan 2022}

\usepackage{amsmath,amssymb}
\usepackage{lmodern}
\usepackage{iftex}
\ifPDFTeX
  \usepackage[T1]{fontenc}
  \usepackage[utf8]{inputenc}
  \usepackage{textcomp} % provide euro and other symbols
\else % if luatex or xetex
  \usepackage{unicode-math}
  \defaultfontfeatures{Scale=MatchLowercase}
  \defaultfontfeatures[\rmfamily]{Ligatures=TeX,Scale=1}
\fi
% Use upquote if available, for straight quotes in verbatim environments
\IfFileExists{upquote.sty}{\usepackage{upquote}}{}
\IfFileExists{microtype.sty}{% use microtype if available
  \usepackage[]{microtype}
  \UseMicrotypeSet[protrusion]{basicmath} % disable protrusion for tt fonts
}{}
\makeatletter
\@ifundefined{KOMAClassName}{% if non-KOMA class
  \IfFileExists{parskip.sty}{%
    \usepackage{parskip}
  }{% else
    \setlength{\parindent}{0pt}
    \setlength{\parskip}{6pt plus 2pt minus 1pt}}
}{% if KOMA class
  \KOMAoptions{parskip=half}}
\makeatother
\usepackage{xcolor}
\IfFileExists{xurl.sty}{\usepackage{xurl}}{} % add URL line breaks if available
\IfFileExists{bookmark.sty}{\usepackage{bookmark}}{\usepackage{hyperref}}
\hypersetup{
  pdftitle={ShareStats},
  hidelinks,
  pdfcreator={LaTeX via pandoc}}
\urlstyle{same} % disable monospaced font for URLs
\usepackage[margin=1in]{geometry}
\usepackage{color}
\usepackage{fancyvrb}
\newcommand{\VerbBar}{|}
\newcommand{\VERB}{\Verb[commandchars=\\\{\}]}
\DefineVerbatimEnvironment{Highlighting}{Verbatim}{commandchars=\\\{\}}
% Add ',fontsize=\small' for more characters per line
\usepackage{framed}
\definecolor{shadecolor}{RGB}{248,248,248}
\newenvironment{Shaded}{\begin{snugshade}}{\end{snugshade}}
\newcommand{\AlertTok}[1]{\textcolor[rgb]{0.94,0.16,0.16}{#1}}
\newcommand{\AnnotationTok}[1]{\textcolor[rgb]{0.56,0.35,0.01}{\textbf{\textit{#1}}}}
\newcommand{\AttributeTok}[1]{\textcolor[rgb]{0.77,0.63,0.00}{#1}}
\newcommand{\BaseNTok}[1]{\textcolor[rgb]{0.00,0.00,0.81}{#1}}
\newcommand{\BuiltInTok}[1]{#1}
\newcommand{\CharTok}[1]{\textcolor[rgb]{0.31,0.60,0.02}{#1}}
\newcommand{\CommentTok}[1]{\textcolor[rgb]{0.56,0.35,0.01}{\textit{#1}}}
\newcommand{\CommentVarTok}[1]{\textcolor[rgb]{0.56,0.35,0.01}{\textbf{\textit{#1}}}}
\newcommand{\ConstantTok}[1]{\textcolor[rgb]{0.00,0.00,0.00}{#1}}
\newcommand{\ControlFlowTok}[1]{\textcolor[rgb]{0.13,0.29,0.53}{\textbf{#1}}}
\newcommand{\DataTypeTok}[1]{\textcolor[rgb]{0.13,0.29,0.53}{#1}}
\newcommand{\DecValTok}[1]{\textcolor[rgb]{0.00,0.00,0.81}{#1}}
\newcommand{\DocumentationTok}[1]{\textcolor[rgb]{0.56,0.35,0.01}{\textbf{\textit{#1}}}}
\newcommand{\ErrorTok}[1]{\textcolor[rgb]{0.64,0.00,0.00}{\textbf{#1}}}
\newcommand{\ExtensionTok}[1]{#1}
\newcommand{\FloatTok}[1]{\textcolor[rgb]{0.00,0.00,0.81}{#1}}
\newcommand{\FunctionTok}[1]{\textcolor[rgb]{0.00,0.00,0.00}{#1}}
\newcommand{\ImportTok}[1]{#1}
\newcommand{\InformationTok}[1]{\textcolor[rgb]{0.56,0.35,0.01}{\textbf{\textit{#1}}}}
\newcommand{\KeywordTok}[1]{\textcolor[rgb]{0.13,0.29,0.53}{\textbf{#1}}}
\newcommand{\NormalTok}[1]{#1}
\newcommand{\OperatorTok}[1]{\textcolor[rgb]{0.81,0.36,0.00}{\textbf{#1}}}
\newcommand{\OtherTok}[1]{\textcolor[rgb]{0.56,0.35,0.01}{#1}}
\newcommand{\PreprocessorTok}[1]{\textcolor[rgb]{0.56,0.35,0.01}{\textit{#1}}}
\newcommand{\RegionMarkerTok}[1]{#1}
\newcommand{\SpecialCharTok}[1]{\textcolor[rgb]{0.00,0.00,0.00}{#1}}
\newcommand{\SpecialStringTok}[1]{\textcolor[rgb]{0.31,0.60,0.02}{#1}}
\newcommand{\StringTok}[1]{\textcolor[rgb]{0.31,0.60,0.02}{#1}}
\newcommand{\VariableTok}[1]{\textcolor[rgb]{0.00,0.00,0.00}{#1}}
\newcommand{\VerbatimStringTok}[1]{\textcolor[rgb]{0.31,0.60,0.02}{#1}}
\newcommand{\WarningTok}[1]{\textcolor[rgb]{0.56,0.35,0.01}{\textbf{\textit{#1}}}}
\usepackage{graphicx}
\makeatletter
\def\maxwidth{\ifdim\Gin@nat@width>\linewidth\linewidth\else\Gin@nat@width\fi}
\def\maxheight{\ifdim\Gin@nat@height>\textheight\textheight\else\Gin@nat@height\fi}
\makeatother
% Scale images if necessary, so that they will not overflow the page
% margins by default, and it is still possible to overwrite the defaults
% using explicit options in \includegraphics[width, height, ...]{}
\setkeys{Gin}{width=\maxwidth,height=\maxheight,keepaspectratio}
% Set default figure placement to htbp
\makeatletter
\def\fps@figure{htbp}
\makeatother
\setlength{\emergencystretch}{3em} % prevent overfull lines
\providecommand{\tightlist}{%
  \setlength{\itemsep}{0pt}\setlength{\parskip}{0pt}}
\setcounter{secnumdepth}{-\maxdimen} % remove section numbering
\ifLuaTeX
  \usepackage{selnolig}  % disable illegal ligatures
\fi

\begin{document}
\maketitle

{
\setcounter{tocdepth}{2}
\tableofcontents
}
\hypertarget{section}{%
\section{}\label{section}}

\begin{figure}
\includegraphics[width=0.2\linewidth]{images/InfShareStatsSquare} \caption{ShareStats logo}\label{fig:unnamed-chunk-3}
\end{figure}

Goal of ShareStats

\hypertarget{outline}{%
\section{Outline}\label{outline}}

\begin{itemize}
\tightlist
\item
  RStudio
\item
  The R/exams package
\item
  Item types
\item
  Adding taxonomy
\item
  Adding tags
\item
  Compiling your item
\item
  Quality check
\end{itemize}

\hypertarget{section-1}{%
\section{}\label{section-1}}

\includegraphics{https://upload.wikimedia.org/wikipedia/commons/thumb/d/d0/RStudio_logo_flat.svg/1280px-RStudio_logo_flat.svg.png}

\hypertarget{installation}{%
\subsection{Installation}\label{installation}}

\begin{itemize}
\tightlist
\item
  \href{https://www.google.com/search?rls=en\&q=how+to+install+r\&ie=UTF-8\&oe=UTF-8}{Install
  R}
\item
  \href{https://www.google.com/search?rls=en\&q=how+to+install+rstudio\&ie=UTF-8\&oe=UTF-8}{Install
  Rstudio}
\end{itemize}

\hypertarget{the-interface}{%
\subsection{The interface}\label{the-interface}}

\begin{itemize}
\tightlist
\item
  \href{https://bookdown.org/gary_a_napier/induction_-_introduction_to_r/rstudio-interface.html}{Rstudio
  basics}
\end{itemize}

\hypertarget{the-rexams-package}{%
\section{The R/exams package}\label{the-rexams-package}}

\begin{figure}
\centering
\includegraphics{http://www.r-exams.org/images/oneforall.svg}
\caption{R/exams}
\end{figure}

\hypertarget{relevant-resources}{%
\subsection{Relevant Resources}\label{relevant-resources}}

\begin{itemize}
\tightlist
\item
  \href{http://www.r-exams.org}{R/exams website}
\item
  \href{http://www.r-exams.org/tutorials/first_steps/}{First Steps}
\end{itemize}

\hypertarget{install-rexams}{%
\subsection{Install R/exams}\label{install-rexams}}

\begin{Shaded}
\begin{Highlighting}[]
\FunctionTok{install.packages}\NormalTok{(}\StringTok{"exams"}\NormalTok{, }\AttributeTok{dependencies =} \ConstantTok{TRUE}\NormalTok{)}
\FunctionTok{library}\NormalTok{(exams)}
\end{Highlighting}
\end{Shaded}

\href{http://www.r-exams.org/resources/}{R/Exams resources page here}.

\hypertarget{procedure}{%
\section{Procedure}\label{procedure}}

Resource: Procedure item development in teams.

\begin{enumerate}
\def\labelenumi{\arabic{enumi}.}
\tightlist
\item
  Go to project folder
\item
  Navigate to desired taxonomy level 1 folder
\item
  Create a new folder for each new item
\item
  Name your item following instruction
\item
  Create HTML version
\item
  Quality check
\end{enumerate}

\hypertarget{naming-instruction}{%
\subsection{Naming instruction}\label{naming-instruction}}

{[}abbreviation institution{]}-{[}lowest taxonomy level for
item{]}-{[}nummber \#\#\#{]}-{[}nl/en{]}

All in small caps.

example:

\begin{itemize}
\tightlist
\item
  uva-regression-001-nl.Rmd
\end{itemize}

If available:

\begin{itemize}
\tightlist
\item
  uva-regression-001-nl-graph01.jpg
\item
  uva-regression-001-nl-data01.sav
\end{itemize}

\hypertarget{item-types}{%
\section{Item types}\label{item-types}}

There are five item types available in R
(\href{http://www.r-exams.org/intro/dynamic/}{Resource}).

\begin{itemize}
\tightlist
\item
  \textbf{Multiple choice:} \texttt{extype:\ schoice} (s: single)
\item
  \textbf{Multiple answer:} \texttt{extype:\ mchoice} (m: multiple)
\item
  \textbf{Fill in the blank numbers:} \texttt{extype:\ num}
\item
  \textbf{Fill in the blank text/essay:} \texttt{extype:\ string}
\item
  \textbf{Combinations:} \texttt{extype:\ cloze}
\item
  Adding images and attachments
\end{itemize}

\hypertarget{general-item-structur}{%
\subsection{General item structur}\label{general-item-structur}}

\begin{verbatim}
Question
========

Solution
========

Meta-information
================
exname: 
extype: 
exsolution: 
exsection: 
exextra[]:
\end{verbatim}

\hypertarget{multiple-choice}{%
\subsection{Multiple choice}\label{multiple-choice}}

\begin{verbatim}
Question
========

What is the average of the numbers 3, 5 and 7?

Answerlist
----------
* 4
* 5
* 6

Solution
========

The correct answer is 5.

Answerlist
----------
* You got it wrong
* Yes you got it
* This is the wrong answer

Meta-information
================
exname: Rekens som
extype: schoice
exsolution: 010
exsection: Descriptive statistics/Summary Statistics/Measures of Location/Mean
exextra[Type]: Calculation
\end{verbatim}

\hypertarget{multiple-answer}{%
\subsection{Multiple answer}\label{multiple-answer}}

\begin{verbatim}
Question
========

Which are parametric tests?

Answerlist
----------
* t-test
* signed-rank test
* ANOVA

Solution
========

The correct answer the ANOVA test

Answerlist
----------
* True. t-test
* False. signed rank test
* True. ANOVA

Meta-information
================
exname: parametric
extype: mchoice
exsolution: 101
exsection: Inferential Statistics/Non-parametric Techniques/Signed Rank test
exextra[Type]: Conceptual, Test choice
exextra[Language]: English
exextra[Level]: Statistical Reasoning
\end{verbatim}

\hypertarget{fill-in-the-blank-number}{%
\subsection{Fill in the blank number}\label{fill-in-the-blank-number}}

\begin{verbatim}
Question
========

What is the average of the numbers 3, 5 and 7?

Solution
========

The correct answer is 5.

Meta-information
================
exname: Rekens som
extype: num
exsolution: 5
extol: 0
exsection: Descriptive statistics/Summary Statistics/Measures of Location/Mean
exextra[Type]: Calculation
\end{verbatim}

\hypertarget{fill-in-the-blank-textessay}{%
\subsection{Fill in the blank
text/essay}\label{fill-in-the-blank-textessay}}

\begin{verbatim}
Question
========

What statistical test do you need to test the difference between two independent groups, assuming all parametric assumptions are met?

Solution
========

The correct answer is independent t-test

Meta-information
================
exname: TestSelectionTtest
extype: string
exsolution: "independent t-test"
exsection: Inferential Statistics/Parametric Techniques/t-test/Independent samples means
exextra[Type]: Test choice
\end{verbatim}

\hypertarget{combinations}{%
\subsection{Combinations}\label{combinations}}

\begin{verbatim}
Question
========

Given are the following numbers: 1, 2, 3, 4, 5, 6. 

Answerlist
----------

* What is the sample mean? 
* What is the estimated population standard deviation?

Solution
========

Answerlist
----------

* The sample mean is 3.5.
* The estimated population standard deviation is 1.87.

Meta-information
================
exname: MeandAndSD
extype: cloze
exclozetype: num|num
exsolution: 3.5|1.870829
extol: 0.01
exextra[Type]: Calculate
\end{verbatim}

\hypertarget{adding-images-and-attachments}{%
\subsection{Adding images and
attachments}\label{adding-images-and-attachments}}

Adding image or files to your question.

\begin{enumerate}
\def\labelenumi{\arabic{enumi}.}
\tightlist
\item
  Add the image or file (.png .jpg .sav etc.) to the same folder as your
  question.
\item
  Add the code below at the top of your .Rmd file.
\item
  Add \texttt{!{[}{]}(myImage.png)} at location for image. Replace
  \texttt{myImage} with your own.
\item
  Add \texttt{{[}Download{]}(myFile.xls)} at location for the download.
  Replace \texttt{myFile} with your own.
\end{enumerate}

\begin{Shaded}
\begin{Highlighting}[]
\InformationTok{\textasciigrave{}\textasciigrave{}\textasciigrave{}\{r, echo = FALSE, results = "hide"\}}
\InformationTok{include\_supplement("myImage.png", recursive = TRUE)}
\InformationTok{\textasciigrave{}\textasciigrave{}\textasciigrave{}}
\end{Highlighting}
\end{Shaded}

\hypertarget{adding-meta-information}{%
\section{Adding meta information}\label{adding-meta-information}}

\hypertarget{statistics-taxonomy}{%
\subsection{Statistics taxonomy}\label{statistics-taxonomy}}

You can add the taxonomy to the \texttt{exsection} attribute in the meta
information.

\begin{verbatim}
Meta-information
================
exsection: Descriptive statistics/Summary Statistics/Measures of Spread/Standard Deviation, Inferential Statistics/Parametric Techniques/t-test/One sample mean
\end{verbatim}

\begin{itemize}
\tightlist
\item
  The
  \href{https://sharestats.github.io/Statistics_Taxonomy/Statistics_Taxonomy.html\#Assigning_taxonomy}{taxonomy
  tree can be found here}.
\item
  Multiple paths can be added by using a comma as seperator.
\item
  Only add path that is most specific for the question.
\item
  \textbf{New lines are not allowed}
\end{itemize}

\hypertarget{adding-tags}{%
\subsection{Adding Tags}\label{adding-tags}}

You can add tags to the \texttt{exextra{[}{]}} attribute in the meta
information.

\begin{verbatim}
Meta-information
================
exextra[Type]: Calculation, Data manipulation
exextra[Program]: SPSS
exextra[Language]: English
exextra[Level]: Statistical Literacy
\end{verbatim}

\begin{itemize}
\tightlist
\item
  We have four categories that can be used
  \href{https://sharestats.github.io/Statistics_Taxonomy/Statistics_Taxonomy.html\#Assigning_taxonomy}{tag
  tab}.
\item
  Multiple tags can be added by using a comma as seperator.
\item
  Type, Language and level are required. Only use Program when needed.
\item
  \textbf{New lines are not allowed}
\end{itemize}

\hypertarget{compiling-your-item}{%
\section{Compiling your item}\label{compiling-your-item}}

\hypertarget{html-example}{%
\subsection{HTML example}\label{html-example}}

You are required to compile the item to HTML. Run the
\texttt{exams2html()} function to test your .Rmd file.

\begin{Shaded}
\begin{Highlighting}[]
\CommentTok{\# install.packages("exams", dependencies = TRUE)}
\FunctionTok{library}\NormalTok{(exams)}
\FunctionTok{exams2html}\NormalTok{(}\AttributeTok{file =} \StringTok{""}\NormalTok{)}
\NormalTok{exams}\SpecialCharTok{:::}\FunctionTok{browse\_exercise}\NormalTok{(}\AttributeTok{file =} \StringTok{""}\NormalTok{)}
\FunctionTok{exams2pdf}\NormalTok{(}\AttributeTok{file =} \StringTok{""}\NormalTok{)}
\FunctionTok{exams2pando}\NormalTok{(}\AttributeTok{file =} \StringTok{""}\NormalTok{)}
\end{Highlighting}
\end{Shaded}

\hypertarget{quality-check}{%
\section{Quality check}\label{quality-check}}

\hypertarget{procedure-1}{%
\subsection{Procedure}\label{procedure-1}}

Follow the quality checklist available in the teams general channel.

\end{document}
